\documentclass[twoside,11pt]{article}

% Any additional packages needed should be included after jmlr2e.
% Note that jmlr2e.sty includes epsfig, amssymb, natbib and graphicx,
% and defines many common macros, such as 'proof' and 'example'.
%
% It also sets the bibliographystyle to plainnat; for more information on
% natbib citation styles, see the natbib documentation, a copy of which
% is archived at http://www.jmlr.org/format/natbib.pdf

\usepackage{jmlr2e}

% Definitions of handy macros can go here
\usepackage{blindtext}

\usepackage{amsmath, amsfonts, amssymb,amscd, enumerate,algorithm,algpseudocode,graphicx,fancyhdr,bbm,tikz,subfigure,listings}
\usepackage[a4paper,bindingoffset=0.2in,left=1in,right=1in,top=1.5in,bottom=1.5in,footskip=.25in]{geometry}
\usetikzlibrary{fit, positioning}
 

\newcommand{\R}{\ensuremath{\mathbb{R}}}
\newcommand{\N}{\ensuremath{\mathbb{N}}}
\newcommand{\Q}{\ensuremath{\mathbb{Q}}}
\newcommand{\Z}{\ensuremath{\mathbb{Z}}}
\newcommand{\I}{\ensuremath{\mathbb{I}}}
\newcommand{\1}{\ensuremath{\mathbbm{1}}}
\newcommand{\calF}{\ensuremath{\mathcal{F}}}
\newcommand{\calB}{\ensuremath{\mathcal{B}}}
\newcommand{\calO}{\ensuremath{\mathcal{O}}}
\newcommand{\E}{\ensuremath{\mathbb{E}}}
\newcommand{\calA}{\ensuremath{\mathcal{A}}}
\newcommand{\calG}{\ensuremath{\mathcal{G}}}
\newcommand{\calJ}{\ensuremath{\mathcal{J}}}
\newcommand{\calT}{\ensuremath{\mathcal{T}}}
\newcommand{\calP}{\ensuremath{\mathcal{P}}}
\newcommand{\calU}{\ensuremath{\mathcal{U}}}
\newcommand{\calC}{\ensuremath{\mathcal{C}}}
\newcommand{\calN}{\ensuremath{\mathcal{N}}}
\newcommand{\calX}{\ensuremath{\mathcal{X}}}
\newcommand{\calY}{\ensuremath{\mathcal{Y}}}
\newcommand{\Cov}{\text{Cov}}
\newcommand{\Var}{\text{Var}}
\newcommand{\MSE}{\text{MSE}}
\newcommand{\logit}{\text{logit}}
\newcommand{\argmax}{\text{argmax}}
\newcommand{\argmin}{\text{argmin}}
\newcommand{\dom}{\text{dom}}
\newcommand{\dist}{\text{dist}}
\newcommand{\prox}{\text{prox}}
\newcommand{\sign}{\text{sign}}
\newcommand{\nuc}{\text{nuc}}
\newcommand{\trace}{\text{trace}}
\newcommand{\<}{\langle}
\renewcommand{\>}{\rangle}
\def\super{\textsuperscript}
\def\sub{\textsubscript}

\newcommand\pderiv[2]{\frac{\partial #1}{\partial #2}}
\newcommand\deriv[2]{\frac{d #1}{d #2}}

\renewcommand{\algorithmicrequire}{\textbf{Input:}}
\renewcommand{\algorithmicensure}{\textbf{Output:}}

% Heading arguments are {volume}{year}{pages}{submitted}{published}{author-full-names}

%\jmlrheading{1}{2000}{1-48}{4/00}{10/00}{meila00a}{Marina Meil\u{a} and Michael I. Jordan}

% Short headings should be running head and authors last names

\ShortHeadings{TREES!!}{TREES!!}
\firstpageno{1}

\begin{document}

\title{TREEES}

\author{\name Eli Ben-Michael \email ebenmichael@berkeley.edu \\
       \addr Department of Statistics\\
       University of California\\
       Berkeley, CA 94720-1776, USA}

\editor{}

\maketitle

\begin{abstract}%   <- trailing '%' for backward compatibility of .sty file
  \blindtext[1]
\end{abstract}

\begin{keywords}
  TREEES
\end{keywords}

\section{Introduction}
\blindtext[1]
\section{Bayesian Optimization and Bandit-Based Algorithms}

\subsection{Bayesian Optimization}

\subsection{SuccessiveHalving and Hyperband}

\section{A Trio of Tree Algorithms}

\subsection{SequentialHalving with Decision Tree inputs}

\subsection{SequentialTree}
\subsubsection{Algorithm}

\begin{algorithm}[H]
\caption{SequentialTree}
\label{seq_tree}
\begin{algorithmic}
\Procedure{SequentialTree}{Box constrained space $\calX \subset \R^d$, budget $T$, $\eta$, $m$}
\State Assign $\calP = \{\calX\}$
\State Assign $\texttt{n\_nodes} = m$
\For{$r =1\ldots \log_\eta(m)$}
\State The number of pulls per element of $\calP$ $n_r = \left \lfloor \frac{T}{|\calP| \lceil \log_\eta(m) \rceil} \right \rfloor$
\State $\calA_r = \emptyset$
\State $\hat{B}_r = \emptyset$ %\Comment{Stores the tree predictions}
\For{$p \in \calP$}
\State Remove $p$ from $\calP$
\State Sample $\{x_{p1},\ldots, x_{pn_r}\}$ independently and uniformly over $p$
\State Let $\{y_{p1},\ldots, y_{pn_r}\}$ be the result of querying the function at these points
\State Train a decision tree with \texttt{n\_nodes} nodes, $t_p = \textproc{DecisionTree}(y_p \sim X_p)$
\State Add the \texttt{n\_nodes} partitions defined by $t_p$ to $\calA_r$
\State Add the prediction from $t_p$ for each partition to $\hat{B}_r$.
\EndFor
\If{$|\calP| > \eta^2$}
\State Set $\calP$ to be the elements of $\calA_r$ with predictions in the top $\frac{|\calP|}{\eta^2}$ of $\hat{B}_r$
\Else 
\State Let $\hat{p}$ be the element of $\calP$ with the best prediction
\State \Return $\textproc{MidPoint}(\hat{p})$
\EndIf
\State $\texttt{n\_nodes} = \eta$
\EndFor

\EndProcedure
\end{algorithmic}

\end{algorithm}
\subsection{PartitionTree}

\subsubsection{Algorithm}

\begin{algorithm}[H]
\caption{PartitionTree}
\label{part_tree}
\begin{algorithmic}
\Procedure{PartitionTree}{Box constrained space $\calX \subset \R^d$, budget $T$, $\eta$, rounds $R$}
\State $m = \eta^2$
\State $\calP = \{\calX\}$
\For{$r=1,\ldots,R$}
The number of pulls per element of $\calP$ $n_r = \left \lfloor \frac{T}{|\calP| R} \right \rfloor$
\State $\calA_r = \emptyset$
\State $\hat{B}_r = \emptyset$ %\Comment{Stores the tree predictions}
\For{$p \in \calP$}
\State Remove $p$ from $\calP$
\State Sample $\{x_{p1},\ldots, x_{pn_r}\}$ independently and uniformly over $p$
\State Let $\{y_{p1},\ldots, y_{pn_r}\}$ be the result of querying the function at these points
\State Train a decision tree with $m$ nodes, $t_p = \textproc{DecisionTree}(y_p \sim X_p)$
\State Add the $m$ partitions defined by $t_p$ to $\calA_r$
\State Add the prediction from $t_p$ for each partition to $\hat{B}_r$.
\EndFor
\State Set $\calP$ to be the elements of $\calA_r$ with predictions in the top $\frac{|\calP|}{\eta}$ of $\hat{B}_r$
\State Let $\hat{p}$ be the element of $\calP$ with the best prediction
\State $m = \eta$
\EndFor
\State \Return $\textproc{MidPoint}(\hat{p})$
\EndProcedure
\end{algorithmic}
\end{algorithm}

\section{Empirical Evaluation}

\section{Conclusion}


% Acknowledgements should go at the end, before appendices and references

\acks{We would like to acknowledge }


\newpage

\appendix
\section*{Appendix A.}

\vskip 0.2in
\bibliography{sample}

\end{document}